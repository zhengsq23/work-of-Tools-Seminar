\documentclass[UTF8]{ctexart}
\usepackage{graphicx}
\usepackage{epstopdf}
\usepackage{amsmath}
\title{数学分析与线性代数例题}
\author{佚名}
\date{\today}
\begin{document}
\maketitle
\tableofcontents
\section{微分中值定理及其应用}
\paragraph{定理 1} (极值的第二充分条件). 设 $f(x)$ 在 $(x_0 -δ,x_0 + δ) $可导且$ {f^{'}}(x_0) = 0$,又 $ {f^{''}}(x_0)$ 存在. \\
1) 若$ {f^{''}}(x_0) < 0 $,则 $f(x_0)$ 是严格极大值\\
2) 若 $ {f^{''}}(x_0) > 0 $,则$ f(x_0) $是严格极小值.\\
\paragraph{例 1.} 求 $y = \frac{1}{3}x \sqrt[3]{(x-5)^2}$的极值点与极值 $\footnote{原题摘自《数学分析简明教程》 (上册)P142.}$.\\
\paragraph{解.} 函数在 $(-\infty,+\infty) $上连续,当$ x \not= 5 $时有${y^{'}}=\frac{1}{3}\Biggl((x-5)^\frac{2}{3}+\frac{2x}{3}(x-5)^{-\frac{1}{3}}\Biggr) = \frac{5(x-3)}{9(x-5)^\frac{1}{3}}.$  \rightline{(1)}\\
令$ (y_0) = 0 $得稳定点 $x = 3$,现列表如下:\\
\begin{tabular}{|c|c|c|c|c|c|}

\hline x& ($-\infty$,3) &3 &(3,5)& 5& (5,$+\infty$)\\

\hline$ y^{'}$& + &0& -& 不存在& + \\

\hline y & $\nearrow$ & $ \sqrt[3]{4}$ &$\searrow$& 0 &$\nearrow$\\

\hline

\end{tabular}\\
从表中可见$ x = 3 $是极大值点,极大值为$ f(3) =\sqrt[3]{4}$ ;$x = 5 $为极小值点,极小值为 $f(5) =0$. 我 们可以大致地画出函数的图形,如图1所示.\\
\begin{figure}
\centerline{\includegraphics[scale=0.5]{function.pdf}}
\caption{$ y = \frac{1}{3}x \sqrt[3]{(x-5)^2}$ 的函数图像 }
\end{figure}

\section{行列式}
\paragraph{例 2.} 若 $a,b \in \textbf{R}$ ,求由方程为$\frac{x_1^2}{a^2}+\frac{x_2^2}{b^2}=1$的椭圆为边界的区域 $E$ 的面积$^2$. 
\paragraph{解.} 断言$ E$ 是单位圆盘 $D$ 在线性变换 $T$ 下的像. 这里 $T$ 由矩阵$A=
\begin{bmatrix} a&0\\0&b \end{bmatrix} \quad$确定,这是因为若$\textbf{u}=\begin{bmatrix} u_1\\u_2 \end{bmatrix}$,$\textbf{x}=\begin{bmatrix} x_1\\x_2 \end{bmatrix}$,且\textbf{x}  = A\textbf{u},则\\
\\
\centerline{$u_1=\frac{x_1}{a},u_2=\frac{x_2}{b}$}\\
从而得  \textbf{u}在此单位圆内,即满足$u_1^2+u_2^2\le1$当且仅当 \textbf{x} 在 $E$ 内,即满足  $(\frac{x_1}{a})^2+(\frac{x_2}{b})^2\le1$. 进而\\
\begin{align}
{a}& = \begin{vmatrix} det A\end{vmatrix} \quad·{D的面积} \\
&= a·b·π·(1)^2\\
&= \pi ab
\end{align}
\end{document}
